% iot_crop_growth.tex - Internet de las Cosas y el crecimiento de cultivos: una revisión sistemática
\documentclass[10pt, twocolumn]{article}

% <PAQUETES>
% Lista de paquetes utilizados
\usepackage[spanish, mexico]{babel}
\usepackage[utf8]{inputenc}
\usepackage[T1]{fontenc}
\usepackage{lmodern}
% </PAQUETES>

\begin{document}

% <TÍTULO>
\title{\textbf{Internet de las Cosas y el crecimiento de cultivos: una revisión sistemática}}
\author{Freddy Íñiguez López\\
	Centro de Investigación en Matemáticas, A.C.,\\
	Zacatecas, México,\\
	\texttt{freddy.iniguez@cimat.mx}}
\date{Diciembre 2015}
\maketitle
% </TÍTULO>

% <ABSTRACT>
\begin{abstract}
Lorem ipsum dolor sit amet, consectetur adipiscing elit. Duis egestas metus ut ante cursus, in gravida elit ultrices. Etiam mi sapien, ornare sit amet vestibulum nec, imperdiet sed orci. Vestibulum ante ipsum primis in faucibus orci luctus et ultrices posuere cubilia Curae; Suspendisse at dui elit. Sed iaculis tempus feugiat. Nam a lorem id orci convallis vestibulum. Cras sollicitudin, dolor nec elementum ornare, neque risus volutpat sem, a venenatis ex quam ut ante. Pellentesque egestas dui sed ex lobortis, nec pharetra diam venenatis. Ut cursus lacinia convallis. Sed ipsum ligula, aliquam at porta vel, feugiat consectetur nunc.
\end{abstract}
% </ABSTRACT>

% <CONTENIDO>
\paragraph{  }
\textbf{Keywords} Internet de las cosas | IoT | Cultivos | Sobrepoblación

\section{INTRODUCCIÓN}
\paragraph{Lorem ipsum dolor sit amet, consectetur adipiscing elit. Duis egestas metus ut ante cursus, in gravida elit ultrices. Etiam mi sapien, ornare sit amet vestibulum nec, imperdiet sed orci. Vestibulum ante ipsum primis in faucibus orci luctus et ultrices posuere cubilia Curae; Suspendisse at dui elit. Sed iaculis tempus feugiat. Nam a lorem id orci convallis vestibulum. Cras sollicitudin, dolor nec elementum ornare, neque risus volutpat sem, a venenatis ex quam ut ante. Pellentesque egestas dui sed ex lobortis, nec pharetra diam venenatis. Ut cursus lacinia convallis. Sed ipsum ligula, aliquam at porta vel, feugiat consectetur nunc.}

\section{REVISIÓN SISTEMÁTICA}
\paragraph{En esta sección se explica con detalle el desarrollo de las actividades de la revisión sistemática, los cuales incluyen el planteamiento de la pregunta que guiará la revisión sistemática, la estrategia a seguir para la recopilación de estudios primarios, la definición de criterios de inclusión y exclusión que serán útiles al momento de realizar el filtro de los estudios primarios.}

\subsection{Antecedentes}
\paragraph{El razonamiento de la revisión.}

\subsection{Preguntas de investigación}
\paragraph{Las preguntas que la investigación pretende responder.}

\subsection{Estrategia}
\paragraph{Estrategia que se utilizará para la búsqueda de estudios primarios incluyendo términos de búsqueda y los recursos en los que serán buscados.}
\paragraph{Los recursos incluyen bibliotecas digitales, revistas específicas y actas de congresos.}

\subsection{Criterios de selección de estudios}
\paragraph{Los critrios de selección de estudios se utilizan para determinar cuáles estudios son incluídos y cuáles son excluídos de la revisión sistemática.}
\paragraph{Es muy útil para pilotear los criterios de selección en un subconjunto de estudios primarios.}

\subsection{Procedimientos de selección de estudios}
\paragraph{El protocolo debe describir cómo serán aplicados los criterios de selección. Por ejemplo, cuántos asesores evaluarán cada estudio primario prospecto.}
\paragraph{Cómo serán resueltos los desacuerdos entre asesores.}

\subsection{Procedimientos y listas de revisión para evaluar la calidad del estudio}
\paragraph{Los investigadores deben desarrollar listas de revisión para evaluar los estudios primarios.}
\paragraph{El propósito de la evaluación de la calidad guiará el desarrollo de las listas de revisión.}

\subsection{Estrategia de extracción de datos}
\paragraph{Ésta define cómo la información requerida de cada estudio primario será obtenida.}
\paragraph{Si los datos requieren manipulación o suposiciones e inferencias, el protocolo debe especificar un proceso de validación apropiado.}

\subsection{Síntesis de datos extraídos}
\paragraph{Definir una estrategia para síntesis de datos.}
\paragraph{Esta estrategia debe aclarar si se pretende aplicar un meta-análisis formal o no, y en caso de aplicarse, qué técnicas se aplicarán.}

\subsection{Estrategia de diseminación}
\paragraph{Para quién serán importantes los resultados de la revisión sistemática.}
\paragraph{Planificar cómo será realizada la difución de los resultados de la revisión sistemática.}

\subsection{Calendarización del proyecto}
\paragraph{Este debe definir el calendario de revisiones.}

\section{METODOLOGÍA DE INVESTIGACIÓN}
\paragraph{Lorem ipsum dolor sit amet, consectetur adipiscing elit. Duis egestas metus ut ante cursus, in gravida elit ultrices. Etiam mi sapien, ornare sit amet vestibulum nec, imperdiet sed orci. Vestibulum ante ipsum primis in faucibus orci luctus et ultrices posuere cubilia Curae; Suspendisse at dui elit. Sed iaculis tempus feugiat. Nam a lorem id orci convallis vestibulum. Cras sollicitudin, dolor nec elementum ornare, neque risus volutpat sem, a venenatis ex quam ut ante. Pellentesque egestas dui sed ex lobortis, nec pharetra diam venenatis. Ut cursus lacinia convallis. Sed ipsum ligula, aliquam at porta vel, feugiat consectetur nunc.}

\section{RESULTADOS}
\paragraph{Lorem ipsum dolor sit amet, consectetur adipiscing elit. Duis egestas metus ut ante cursus, in gravida elit ultrices. Etiam mi sapien, ornare sit amet vestibulum nec, imperdiet sed orci. Vestibulum ante ipsum primis in faucibus orci luctus et ultrices posuere cubilia Curae; Suspendisse at dui elit. Sed iaculis tempus feugiat. Nam a lorem id orci convallis vestibulum. Cras sollicitudin, dolor nec elementum ornare, neque risus volutpat sem, a venenatis ex quam ut ante. Pellentesque egestas dui sed ex lobortis, nec pharetra diam venenatis. Ut cursus lacinia convallis. Sed ipsum ligula, aliquam at porta vel, feugiat consectetur nunc.}

\section{CONCLUSIONES}
\paragraph{Una vez obtenidas las condiciones climáticas óptimas de determinado cultivo, se propone desarrollar un modelos probabilístico con el cuál se pueda llegar a predecir o dar una estimación muy certera acerca de los cuidados y acciones a implementar para que las condiciones del cultivo se mantengan y su producción no se vea afectada por algún cambio climático.}

\section*{Referencias}
\paragraph{Lorem ipsum dolor sit amet, consectetur adipiscing elit. Duis egestas metus ut ante cursus, in gravida elit ultrices. Etiam mi sapien, ornare sit amet vestibulum nec, imperdiet sed orci. Vestibulum ante ipsum primis in faucibus orci luctus et ultrices posuere cubilia Curae; Suspendisse at dui elit. Sed iaculis tempus feugiat. Nam a lorem id orci convallis vestibulum. Cras sollicitudin, dolor nec elementum ornare, neque risus volutpat sem, a venenatis ex quam ut ante. Pellentesque egestas dui sed ex lobortis, nec pharetra diam venenatis. Ut cursus lacinia convallis. Sed ipsum ligula, aliquam at porta vel, feugiat consectetur nunc.}

\end{document}
% </CONTENIDO>


% --- Report structure ---
% Title
% Authorship
% Executive summary or structured abstract
% 	- Context
% 	- Objectives
% 	- Methods
% 	- Results
% 	- Conclusions
% Background
% Review questions
% Review method
% 	- Data sources and search strategy
% 	- Study selection
% 	- Study quality assessment
% 	- Data extraction
% 	- Data synthesis
% Included and excluded studies
% Results
% 	- Findings
% 	- Sensitivy analysis
% Discussion
% 	- Principal findings
% 	- Strenghts and weakness
% 	- Meaning of findings
% Conclusions
% 	- Recommendations
% Acknowledgements
% Conflict of interest
% References and appendices

% chain:
% Internet of Things OR IoT AND crop growth
% Internet of Things OR IoT AND agriculture
