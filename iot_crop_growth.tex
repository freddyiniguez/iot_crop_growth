% iot_crop_growth.tex - Internet de las Cosas y el crecimiento de cultivos: una revisión sistemática
\documentclass[10pt, twocolumn]{article}

% ----------------------------------------------------- PAQUETES ----------------------------------------------------- %
% Lista de paquetes utilizados
\usepackage[spanish, mexico]{babel}
\usepackage[utf8]{inputenc}
\usepackage[T1]{fontenc}
\usepackage{lmodern}

\begin{document}

% ----------------------------------------------------- TÍTULO ----------------------------------------------------- %
\title{\textbf{Internet de las Cosas y el crecimiento de cultivos: una revisión sistemática}}
\author{Freddy Íñiguez López\\
	Centro de Investigación en Matemáticas, A.C.,\\
	Zacatecas, México,\\
	\texttt{freddy.iniguez@cimat.mx}}
\date{Diciembre 2015}
\maketitle

% ----------------------------------------------------- ABSTRACT ----------------------------------------------------- %
\begin{abstract}
% 	- Context
% 	- Objectives
% 	- Methods
% 	- Results
% 	- Conclusions
\end{abstract}
\paragraph{  }
\textbf{Keywords} Internet de las Cosas | IoT | Cultivos | Sobrepoblación

% ----------------------------------------------------- INTRODUCCIÓN ----------------------------------------------------- %
\section{INTRODUCCIÓN}
\paragraph{[En desarrollo.]}

\paragraph{Los resultados obtenidos de esta revisión sistemática pretenden servir como base para la investigación de técnicas de Internet de las Cosas que ayuden a determinar las condiciones climáticas y de crecimiento para cierto cultivo y posteriormente desarrollar un modelo probabilístico que ayude a los agricultores y empresas de cultivos a mantener las mejores condiciones para un crecimiento óptimo y apresurado de sus cultivos.}

% ----------------------------------------------------- REVISIÓN SISTEMÁTICA ----------------------------------------------------- %
\section{REVISIÓN SISTEMÁTICA}
\paragraph{Esta sección contiene el detalle de las actividades desarrolladas correspondientes a la fase de planeación de la revisión sistemática. De tal manera, se muestra el planteamiento de la pregunta de investigación, en torno a la cual giran todas las demás actividades de la revisión sistemática, se detalla la estrategia a seguir para la recopilación de los estudios primarios y se muestra la definición de los criterios de inclusión y exclusión, los cuáles serán útiles al momento de realizar el filtro de los estudios primarios.}

\subsection{Antecedentes}
% El razonamiento de la revisión
\paragraph{De acuerdo con \cite{globalsustainabilityreport}, para el año 2030 se estima que la población demando por lo menos un 50 por ciento más de comida, un 30 por ciento más de agua.}

\subsection{Preguntas de investigación}
% Las preguntas que la investigación pretende responder.

\subsection{Estrategia}
% Estrategia que se utilizará para la búsqueda de estudios primarios incluyendo términos de búsqueda y los recursos en los que serán buscados.
% Los recursos incluyen bibliotecas digitales, revistas específicas y actas de congresos.
\paragraph{}
% Términos de búsqueda:
% Internet of Things OR IoT AND crop growth
% Internet of Things OR IoT AND agriculture
% Internet of Things Or IoT AND farm

\subsection{Criterios de selección de estudios}
% Los critrios de selección de estudios se utilizan para determinar cuáles estudios son incluídos y cuáles son excluídos de la revisión sistemática.
% Es muy útil para pilotear los criterios de selección en un subconjunto de estudios primarios.
\paragraph{}

\subsection{Procedimientos de selección de estudios}
% El protocolo debe describir cómo serán aplicados los criterios de selección. Por ejemplo, cuántos asesores evaluarán cada estudio primario prospecto.
% Cómo serán resueltos los desacuerdos entre asesores.
\paragraph{}

\subsection{Procedimientos y listas de revisión para evaluar la calidad del estudio}
% Los investigadores deben desarrollar listas de revisión para evaluar los estudios primarios.
% El propósito de la evaluación de la calidad guiará el desarrollo de las listas de revisión.
\paragraph{}

\subsection{Estrategia de extracción de datos}
% Ésta define cómo la información requerida de cada estudio primario será obtenida.
% Si los datos requieren manipulación o suposiciones e inferencias, el protocolo debe especificar un proceso de validación apropiado.
\paragraph{}

\subsection{Síntesis de datos extraídos}
% Definir una estrategia para síntesis de datos.
% Esta estrategia debe aclarar si se pretende aplicar un meta-análisis formal o no, y en caso de aplicarse, qué técnicas se aplicarán.
\paragraph{}

\subsection{Estrategia de diseminación}
% Para quién serán importantes los resultados de la revisión sistemática.
% Planificar cómo será realizada la difución de los resultados de la revisión sistemática.
\paragraph{}

\subsection{Calendarización del proyecto}
% Este debe definir el calendario de revisiones
\paragraph{}

% ----------------------------------------------------- METODOLOGÍA DE INVESTIGACIÓN ----------------------------------------------------- %
\section{METODOLOGÍA DE INVESTIGACIÓN}
\paragraph{}
% Review method
% 	- Data sources and search strategy
% 	- Study selection
% 	- Study quality assessment
% 	- Data extraction
% 	- Data synthesis
% Included and excluded studies

% ----------------------------------------------------- RESULTADOS ----------------------------------------------------- %
\section{RESULTADOS}
\paragraph{}
% Results
% 	- Findings
% 	- Sensitivy analysis
% Discussion
% 	- Principal findings
% 	- Strenghts and weakness
% 	- Meaning of findings

% ----------------------------------------------------- CONCLUSIONES ----------------------------------------------------- %
\section{CONCLUSIONES}
\paragraph{}
% Conclusions
% 	- Recommendations

% ----------------------------------------------------- REFERENCIAS ----------------------------------------------------- %
\section*{Referencias}
\bibliographystyle{plain}
\bibliography{iot_crop_growth}

\end{document}

% Bernand Marr:
% "And in agriculture, data analysis is helping the industry meet the challenge of increasing the world's food production by 60%, as forecasters have said will be necessary by 2050 due to the growing population. Tractor and agricultural machinery manufacturer, John Deere, already fits sensors to its machinery. The data that is available to the farmers via its myjohndeere.com and Farmsight services helps them to establish optimum conditions for their crops. Plus the data is also used by John Deere to forecast demand for spare parts."