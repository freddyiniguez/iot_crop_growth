% iot_crop_growth.tex - Internet de las Cosas y el crecimiento apresurado de cultivos: una revisión sistemática
\documentclass[10pt, twocolumn]{article}

% ----------------------------------------------------- PAQUETES ----------------------------------------------------- %
% Lista de paquetes utilizados
\usepackage[spanish, mexico]{babel}
\usepackage[utf8]{inputenc}
\usepackage[T1]{fontenc}
\usepackage{lmodern}

\begin{document}

% ----------------------------------------------------- TÍTULO ----------------------------------------------------- %
% Propuestas: 
% - Internet de las Cosas y el crecimiento de cultivos: una revisión sistemática
% - Internet de las Cosas y el crecimiento apresurado de cultivos: una revisión sistemática
% - Internet de las Cosas y la agricultura: una revisión sistemática
\title{\textbf{Internet de las Cosas y el crecimiento apresurado de cultivos: una revisión sistemática - BORRADOR}}
\author{Freddy Íñiguez López\\
	Centro de Investigación en Matemáticas, A.C.,\\
	Zacatecas, México,\\
	\texttt{freddy.iniguez@cimat.mx}}
\date{Diciembre 2015}
\maketitle

% ----------------------------------------------------- ABSTRACT ----------------------------------------------------- %
\begin{abstract}
% 	- Context
% 	- Objectives
% 	- Methods
% 	- Results
% 	- Conclusions
[En desarrollo]
\end{abstract}
\paragraph{}
\textbf{Keywords} Internet de las Cosas | IoT | Agricultura | Sobrepoblación

% ----------------------------------------------------- INTRODUCCIÓN ----------------------------------------------------- %
\section{INTRODUCCIÓN}
\paragraph{[En desarrollo] \\ Los resultados obtenidos de esta revisión sistemática pretenden servir como base para la investigación de técnicas de Internet de las Cosas que ayuden a determinar las condiciones climáticas y de crecimiento para cierto cultivo y posteriormente desarrollar un modelo probabilístico que ayude a los agricultores y empresas de cultivos a mantener las mejores condiciones para un crecimiento óptimo y apresurado de sus cultivos.}

% ----------------------------------------------------- REVISIÓN SISTEMÁTICA ----------------------------------------------------- %
% Etapa de la planificación de la revisión sistemática.
\section{REVISIÓN SISTEMÁTICA}
\paragraph{Esta sección contiene el detalle de las actividades desarrolladas correspondientes a la fase de planeación de la revisión sistemática. De tal manera, se muestra el planteamiento de la pregunta de investigación, en torno a la cual giran todas las demás actividades de la revisión sistemática, se detalla la estrategia a seguir para la recopilación de los estudios primarios y se muestra la definición de los criterios de inclusión y exclusión, los cuáles serán útiles al momento de realizar el filtro de los estudios primarios.}

\subsection{Antecedentes}
% El razonamiento de la revisión
\paragraph{De acuerdo con el reporte del panel para la sustentabilidad global de la Secretaría General de las Naciones Unidas, para el año 2030 la población requerirá de por lo menos 50\% más comida, 45\% más energía y 30\% más agua~\cite{globalsustainabilityreport}. Específicamente para las condiciones alimentícias a futuro, se estima que para el año 2050 la demanda de comida crezca a un 70 por ciento. \\ Además, cabe mencionar que las condiciones climáticas que vivimos hoy en día son producto del cambio climático y del calentamiento global, problemas ambientales que si se presentan en condiciones extremas pueden llegar a afectar a la mayoría de los cultivos, empeorando la situación del suministro de comida.}

\paragraph{Por tal razón, es de vital importancia que se desarrollen estrategias que ayuden a combatir esta serie de problemas que se preveen para el futuro. En el caso de incrementar los suministros de comida, se deben implementar estrategias para que los campos de cultivo generen una mayor cantidad de alimento, o que puedan generar la misma cantidad de alimento pero tener más temporadas de producción al año.}

\subsection{Pregunta de investigación}
% Las preguntas que la investigación pretende responder.
% El objetivo de una revisión sistemática es encontrar la mayor cantidad de estudios primarios como sea posible, relacionados con una pregunta de investigación usando una estrategia de búsqueda imparcial.
\paragraph{Con la intención de conocer las estrategias, métodos o procedimientos que existen en la actualidad para determinar las características ambientales y de crecimiento de los cultivos, se desarrolla la siguiente pregunta de investigación, la cual a su vez servirá de guía para la definición de las demás actividades de la revisión sistemática.}
% Población: granjas y agricultores.
% Intervención: Tecnologías/herramientas/métodos/metodologías de Internet de las Cosas.
% Comparación: las condiciones climáticas para el crecimiento óptimo de cultivos.
% Salidas: ayudar a tener condiciones óptimas en los cultivos.
% Contexto:
\begin{itemize}
	\item{¿Qué herramientas, estrategias o métodos enfocados al Internet de las Cosas existen para determinar las condiciones óptimas de los cultivos que ayuden a incrementar o a acelerar la producción de los mismos?}
\end{itemize}

\subsection{Estrategia}
% Estrategia que se utilizará para la búsqueda de estudios primarios incluyendo términos de búsqueda y los recursos en los que serán buscados.
% Los recursos incluyen bibliotecas digitales, revistas específicas y actas de congresos.
\paragraph{Para la búsqueda de estudios primarios es necesario definir las cadenas o términos de búsqueda así como también las fuentes en donde serán aplicadas estas búsquedas, como revistas científicas, bases de datos, libros electrónicos u otros recursos libres y bibliotecas. \\ De tal manera, se muestran a continuación las cadenas de búsqueda a aplicar.}
\begin{itemize}
	\item{Internet of Things OR IoT AND crop growth}
	\item{Internet of Things OR IoT AND agriculture}
	\item{Internet of Things OR IoT AND food production}
	\item{Internet of Things OR IoT AND rapid crop growth}
	\item{Internet of Things OR IoT AND accelerated crop growth}
\end{itemize}
\paragraph{Y para el caso de las fuentes que serán utilizadas se muestran a continuación aquellas seleccionadas por su alto grado de relevancia en el campo científico.}
\begin{itemize}
	\item{ACM Digital Library}
	\item{IEEE Explore}
	\item{SpringerLink}
\end{itemize}

\subsection{Criterios de selección de estudios}
% Los criterios de selección de estudios se utilizan para determinar cuáles estudios son incluídos y cuáles son excluídos de la revisión sistemática.
% Es muy útil para pilotear los criterios de selección en un subconjunto de estudios primarios.
\paragraph{Una vez que se apliquen las cadenas de búsqueda y se obtenga un conjunto de estudios, será necesario realizar una evaluación a cada uno y categorizarlos de acuerdo al tipo de estudio del que se trate. Para ello, se deben definir criterios para la selección de estudios, que para el caso de la presente revisión sistemática se detallan a continuación.}
% Criterios de inclusión
\paragraph{Se incluirán en la revisión sistemática aquellos estudios que sean o que cumplan con las siguientes características:}
\begin{itemize}
	\item{Técnicas, métodos o herramientas de Internet de las Cosas para medir las condiciones climáticas de un cultivo.}
	\item{Aplicaciones del Internet de las Cosas en la agricultura.}
	\item{Sistemas o aplicaciones de producción agrícola basado en Internet de las Cosas.}
	\item{Sistemas o aplicaciones de distribución y producción de cultivos basado en Internet de las Cosas.}
	\item{Ecosistemas de agricultura basado en Internet de las Cosas.}
	\item{Sistemas o aplicaciones de monitoreo de cultivos basado en Internet de las Cosas.}
	\item{Sistemas de trazabilidad de cultivos basado en Internet de las Cosas.}
	\item{Marcos de trabajo para el monitoreo de cultivos basado en Internet de las Cosas.}
	\item{Plataformas de monitoreo y control de cultivos basado en Internet de las Cosas.}
	\item{Modelos de aceleración de cultivos basado en Internet de las Cosas.}	
	\item{Investigaciones acerca de sistemas inteligentes de granjas basado en Internet de las Cosas.}
	\item{Sistemas o aplicaciones para la detección temprana de anomalías en el crecimiento de los cultivos.}
	\item{Sistemas o aplicaciones para la detección temprana de anomalías en el crecimiento de una especie determinada de cultivo.}
	\item{Investigaciones o sistemas inteligentes de granjas o \textit{smart farm} para la detección temprana de anomalías en el crecimiento de una especie determinada de cultivo.}
\end{itemize}
% Criterios de exclusión
\paragraph{Se excluirán de la revisión sistemática aquellos estudios que sean o que cumplan las siguientes características:}
\begin{itemize}
	\item{Técnicas, métodos o herramientas que midan las condiciones climáticas de un cultivo pero que éstas no se relacionen con técnicas de Internet de las Cosas.}
	\item{Aplicaciones del Internet de las Cosas para la gestión y control de sistemas agrícolas.}
	\item{Sistemas de producción agrícola tradicionales.}
	\item{Sistemas o aplicaciones de distribución y producción agrícola tradicionales.}
	\item{Investigaciones sobre sistemas inteligentes de granjas o \textit{smart farm} para la detección temprana de plagas.}
	\item{Investigaciones sobre Internet de las Cosas.}
	\item{Investigaciones sobre agricultura.}
\end{itemize}

\subsection{Procedimientos de selección de estudios}
% El protocolo debe describir cómo serán aplicados los criterios de selección. Por ejemplo, cuántos asesores evaluarán cada estudio primario prospecto.
% Cómo serán resueltos los desacuerdos entre asesores.
\paragraph{[En desarrollo]}

\subsection{Procedimientos y listas de revisión para evaluar la calidad del estudio}
% Los investigadores deben desarrollar listas de revisión para evaluar los estudios primarios.
% El propósito de la evaluación de la calidad guiará el desarrollo de las listas de revisión.
\paragraph{[En desarrollo]}

\subsection{Estrategia de extracción de datos}
% Ésta define cómo la información requerida de cada estudio primario será obtenida.
% Si los datos requieren manipulación o suposiciones e inferencias, el protocolo debe especificar un proceso de validación apropiado.
\paragraph{[En desarrollo]}

\subsection{Síntesis de datos extraídos}
% Definir una estrategia para síntesis de datos.
% Esta estrategia debe aclarar si se pretende aplicar un meta-análisis formal o no, y en caso de aplicarse, qué técnicas se aplicarán.
\paragraph{[En desarrollo]}

\subsection{Estrategia de diseminación}
% Para quién serán importantes los resultados de la revisión sistemática.
% Planificar cómo será realizada la difución de los resultados de la revisión sistemática.
\paragraph{[En desarrollo]}

\subsection{Calendarización del proyecto}
% Este debe definir el calendario de revisiones
\paragraph{[En desarrollo]}

% ----------------------------------------------------- METODOLOGÍA DE INVESTIGACIÓN ----------------------------------------------------- %
% Etapa de la ejecución de la revisión sistemática.
\section{METODOLOGÍA DE INVESTIGACIÓN}
\paragraph{[En desarrollo]}
% Review method
% 	- Data sources and search strategy
% 	- Study selection
% 	- Study quality assessment
% 	- Data extraction
% 	- Data synthesis
% Included and excluded studies

% ----------------------------------------------------- RESULTADOS ----------------------------------------------------- %
% Etapa de reportar resultados de la revisión sistemática.
\section{RESULTADOS}
\paragraph{[En desarrollo]}
% Results
% 	- Findings
% 	- Sensitivy analysis
% Discussion
% 	- Principal findings
% 	- Strenghts and weakness
% 	- Meaning of findings

% ----------------------------------------------------- CONCLUSIONES ----------------------------------------------------- %
\section{CONCLUSIONES}
\paragraph{[En desarrollo]}
% Conclusions
% 	- Recommendations

% ----------------------------------------------------- REFERENCIAS ----------------------------------------------------- %
\bibliographystyle{plain}
\bibliography{iot_crop_growth}

\end{document}